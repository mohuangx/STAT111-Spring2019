\documentclass[10pt, handout, xcolor=table]{beamer}

\usepackage[utf8]{inputenc}
\usepackage{amsmath}
\usepackage{amsfonts}
\usepackage{amssymb}
\usepackage{mathtools}
\newcommand*\themecol{\usebeamercolor[fg]{structure}}

\setbeamertemplate{navigation symbols}{}
 \setbeamertemplate{footline}[frame number]

\newcommand{\overbar}[1]{\mkern 1.5mu\overline{\mkern-1.5mu#1\mkern-1.5mu}\mkern 1.5mu}

\usepackage{tikz}
\usetikzlibrary{shapes.geometric, arrows}
\tikzstyle{prob} = [rectangle, minimum width=3cm, text width = 4.5cm, minimum height=1cm, text centered, draw=black, fill= blue!20]
\tikzstyle{stat} = [rectangle, minimum width=3cm,  text width = 4.5cm, minimum height=1cm, text centered, draw=black, fill= red!20]
\tikzstyle{arrow} = [thick,->,>=stealth]

\DeclarePairedDelimiter\abs{\lvert}{\rvert}%
\DeclarePairedDelimiter\norm{\lVert}{\rVert}
\DeclarePairedDelimiter\ceil{\lceil}{\rceil}
\DeclarePairedDelimiter\floor{\lfloor}{\rfloor}



\setlength{\parindent}{0pt}
\setlength{\parskip}{6pt}


\title{STAT 111\\
{\small Recitation 10}}

\author{Mo Huang}
\institute{Email: mohuang@wharton.upenn.edu \\
\vspace{0.25cm}
Office Hours: Wednesdays 3:00 - 4:00 pm, JMHH F96\\
\vspace{0.25cm}
Slides: \url{github.com/mohuangx/STAT111-Spring2019} }


\date{April 12, 2019}


\begin{document}

\begin{frame}
\titlepage
\end{frame}

\begin{frame}{Two-sample $t$-test}
\begin{itemize}
\setlength{\itemsep}{8pt}
\item Example: Suppose we have two classes, let's call 201 and 202, and we want to see if there is a difference in height between these two classes. How do we test this?
\item<2-> Answer: Let $X_{11}, \dots, X_{1m}$ represent the heights of the $m$ students in 201 with mean $\mu_1$ and let $X_{21}, \dots, X_{2n}$ represent the heights of the $n$ students in 202 with mean $\mu_2$. We can test $H_0: \mu_1 = \mu_2$ against $H_1: \mu_1 \neq \mu_2$.
\item<3-> This is an example of a two-sample $t$-test!
\item<3-> Two-sample $t$-test: Let $X_{11}, \dots, X_{1m}$ be i.i.d random variables with (unknown) mean $\mu_1$ and (unknown) variance $\sigma^2$. Let $X_{21}, \dots, X_{2n}$ be i.i.d random variables with (unknown) mean $\mu_2$ and (unknown) variance $\sigma^2$. We want to test whether or not $\mu_1 = \mu_2$.
\end{itemize}
\end{frame}

\begin{frame}{Example}
\begin{itemize}
\setlength{\itemsep}{8pt}
\item Suppose we want to test whether there is a difference in height between class 201 ($X_{11}, \dots, X_{1m}$)  and 202 ($X_{21}, \dots, X_{2n}$). We observe that $\overbar{x}_1 = 66.7$, $s_1^2 = 10.5$, $m = 28$, and $\overbar{x}_2 = 65.6$, $s_2^2 = 12.3$, $n = 34$. 
\item<1->[Step 1] $H_0: \mu_1 = \mu_2$ vs. $H_1: \mu_1 \neq \mu_2$. Two-sided test.   
\item<2->[Step 2] Choose $\alpha = 0.05$. 
\item<3->[Step 3] Test-statistic is 
\[
t = \frac{\overbar{x}_1-\overbar{x}_2}{s\sqrt{\frac{1}{m} + \frac{1}{n}}}, \text{ where } s = \sqrt{\frac{(m-1)s_1^2 + (n-1)s_2^2}{m+n-2}}
\]
\item<4->[]
\[
t = \frac{66.7-65.6}{3.390\sqrt{\frac{1}{28} + \frac{1}{34}}} = 1.254
\]
\end{itemize}
\end{frame}

\begin{frame}{Example}
\begin{itemize}
\setlength{\itemsep}{8pt}
\item Suppose we want to test whether there is a difference in height between class 201 ($X_{11}, \dots, X_{1m}$)  and 202 ($X_{21}, \dots, X_{2n}$). We observe that $\overbar{x}_1 = 66.7$, $s_1^2 = 10.5$, $m = 28$, and $\overbar{x}_2 = 65.6$, $s_2^2 = 12.3$, $n = 34$. 
\item[Step 1] $H_0: \mu_1 = \mu_2$ vs. $H_1: \mu_1 \neq \mu_2$. Two-sided test.   
\item[Step 2] Choose $\alpha = 0.05$. 
\item[Step 3] Test-statistic is $t = 1.254$.
\item[Step 4] Find the critical region.
\item<2->[] How many degrees of freedom do we have? \only<3->{$m+n-2 = 60$}
\item<4->[] So we need to look at $t_{60}$. What is the critical region?
\item<5->[] $t \geq t_{60, 0.025} = 2.000$ and $t \leq -t_{60, 0.025} = -2.000$.
\end{itemize}
\end{frame}

\begin{frame}{Example}
\begin{itemize}
\setlength{\itemsep}{12pt}
\item Suppose we want to test whether there is a difference in height between class 201 ($X_{11}, \dots, X_{1m}$)  and 202 ($X_{21}, \dots, X_{2n}$). We observe that $\overbar{x}_1 = 66.7$, $s_1^2 = 10.5$, $m = 28$, and $\overbar{x}_2 = 65.6$, $s_2^2 = 12.3$, $n = 34$. 
\item[Step 1] $H_0: \mu_1 = \mu_2$ vs. $H_1: \mu_1 \neq \mu_2$. Two-sided test.   
\item[Step 2] Choose $\alpha = 0.05$. 
\item[Step 3] Test-statistic is $t = 1.254$.
\item[Step 4] Find the critical region: $t \geq 2.000$ and $t \leq -2.000$
\item[Step 5] Do we reject $H_0$? \only<2->{No, $t = 1.254$ is not in the critical region.}
\end{itemize}
\end{frame}

\begin{frame}{Paired two sample $t$ test}
\begin{itemize}\itemsep2ex
\item Suppose we have two samples where there is a natural pairing of data between the two samples. Let $\mu_d$ be the mean difference between the two samples. 
\item For example, we have $n$ patients and we are interested in determining if a drug decreases cholesterol levels. We collect cholesterol levels before ($x_{11}, \ldots, x_{1n}$) and after ($x_{21}, \ldots, x_{2n}$) administering the drug. 
\item We want to test $H_0: \mu_d = 0$.
\item Consider $d_i = x_{2i} - x_{1i}$, the difference in measurement between sample 2 and sample 1 for subject $i$.
\begin{itemize}
\item Estimate of $\mu_d$: $\overbar{d} = \frac{1}{n} \sum_{i=1}^n d_i$
\item Estimate of $\sigma^2$: $s^2_d = \frac{d_1^2 + d_2^2 + \cdots + d_n^2 - n(\overbar{d})^2}{n-1}$
\end{itemize}
\item The test statistic is
{\themecol
\[
t = \frac{\overbar{d}}{s_d/\sqrt{n}}
\]
}
\end{itemize}
\end{frame}

\begin{frame}{Example}
\begin{itemize}
\setlength{\itemsep}{8pt}
\item Suppose we have $10$ patients and we are interested in determining if a drug decreases cholesterol levels. We collect the following cholesterol levels before and after administering the drug:
{\scriptsize 
\begin{table}[]
\begin{tabular}{|c|c|c|c|c|c|c|c|c|c|c|}
\hline
Patient    & 1   & 2   & 3   & 4   & 5   & 6   & 7   & 8   & 9   & 10  \\ \hline
Before     & 204 & 243 & 253 & 212 & 239 & 241 & 256 & 267 & 231 & 251 \\ \hline
After      & 200 & 235 & 256 & 200 & 232 & 210 & 249 & 270 & 233 & 243 \\ \hline
Difference & -4  & -8  & 3   & -12 & -7  & -31 & -7  & 3   & 2   & -8  \\ \hline
\end{tabular}
\end{table}
}
\item<1->[Step 1] $H_0: \mu_d = 0$ vs. $H_1: \mu_d < 0$. One-sided test. 
\item<2->[Step 2] Choose $\alpha = 0.01$. 
\item<3->[Step 3] Test-statistic is 
{\small
\[
t = \frac{\overbar{d}}{s_d/\sqrt{n}}
\]
}
\item<4->[]
{\small
\begin{align*}
\only<4->{\overbar{d} &= -6.9 \quad \quad s_d = 9.96\\}
\only<5->{t &= \frac{\overbar{d}}{s_d/\sqrt{n}} = \frac{-6.9}{9.96/\sqrt{10}} = -2.191}
\end{align*}
}
\end{itemize}
\end{frame}

\begin{frame}{Example}
\begin{itemize}
\setlength{\itemsep}{8pt}
\item Suppose we have $10$ patients and we are interested in determining if a drug decreases cholesterol levels. We collect the following cholesterol levels before and after administering the drug:
{\scriptsize 
\begin{table}[]
\begin{tabular}{|c|c|c|c|c|c|c|c|c|c|c|}
\hline
Patient    & 1   & 2   & 3   & 4   & 5   & 6   & 7   & 8   & 9   & 10  \\ \hline
Before     & 204 & 243 & 253 & 212 & 239 & 241 & 256 & 267 & 231 & 251 \\ \hline
After      & 200 & 235 & 256 & 200 & 232 & 210 & 249 & 270 & 233 & 243 \\ \hline
Difference & -4  & -8  & 3   & -12 & -7  & -31 & -7  & 3   & 2   & -8  \\ \hline
\end{tabular}
\end{table}
}
\item[Step 1] $H_0: \mu_d = 0$ vs. $H_1: \mu_d < 0$. One-sided test.    
\item[Step 2] Choose $\alpha = 0.01$. 
\item[Step 3] Test-statistic is $t = -2.191$.
\item[Step 4] Find the critical region.
\item<2->[] How many degrees of freedom do we have? \only<3->{$n-1 = 9$}
\item<4->[] So we need to look at $t_{9}$. What is the critical region?
\item<5->[] $t \leq -t_{9, 0.01} = -2.821$.
\end{itemize}
\end{frame}

\begin{frame}{Example}
\begin{itemize}
\setlength{\itemsep}{12pt}
\item Suppose we have $10$ patients and we are interested in determining if a drug decreases cholesterol levels. We collect the following cholesterol levels before and after administering the drug:
{\scriptsize 
\begin{table}[]
\begin{tabular}{|c|c|c|c|c|c|c|c|c|c|c|}
\hline
Patient    & 1   & 2   & 3   & 4   & 5   & 6   & 7   & 8   & 9   & 10  \\ \hline
Before     & 204 & 243 & 253 & 212 & 239 & 241 & 256 & 267 & 231 & 251 \\ \hline
After      & 200 & 235 & 256 & 200 & 232 & 210 & 249 & 270 & 233 & 243 \\ \hline
Difference & -4  & -8  & 3   & -12 & -7  & -31 & -7  & 3   & 2   & -8  \\ \hline
\end{tabular}
\end{table}
}
\item[Step 1] $H_0: \mu_d = 0$ vs. $H_1: \mu_d < 0$. One-sided test.       
\item[Step 2] Choose $\alpha = 0.01$. 
\item[Step 3] Test-statistic is $t = -2.191$.
\item[Step 4] Find the critical region: $t \leq -2.821$
\item[Step 5] Do we reject $H_0$? \only<2->{No, $t = -2.191$ is not in the critical region.}
\end{itemize}
\end{frame}




\end{document}


